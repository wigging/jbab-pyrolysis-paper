%!TEX root = main.tex

\section{Modeling approach}

Engineering correlations, reduced-order models, and CFD modeling techniques were used to investigate the effects of recycled gas on the operation of a fluidized-bed biomass pyrolysis reactor. The following sections discuss approaches implemented in this work for calculating gas properties and the associated effects on fluidization conditions and pyrolysis yields.

\subsection{Gas properties}

Density (kg/m$^3$) of an individual gas is calculated from the ideal gas law

\begin{equation}
    \rho_{gas} = \frac{P\,M}{R\,T}
\end{equation}

\noindent where $P$ is pressure (Pa), $M$ is molecular weight (g/mol), $R$ is the gas constant [(m$^3$ Pa) / (K mol)], and $T$ is temperature (K). Gas viscosity (\textmugreek P) is given as

\begin{equation}
    \mu_{gas} = A + B\,T + C\,T^2 + D\,T^3
\end{equation}

\noindent where coefficients $A$, $B$, $C$, and $D$ are obtained for a given gas from tables in Yaws' Handbook and $T$ is gas temperature (K). Thermal conductivity ($k$) of the gas is given as

\begin{equation}
    k_{gas} = A + B\,T + C\,T^2 + D\,T^3
\end{equation}

\noindent where coefficients $A$, $B$, $C$, and $D$ are also from the Yaws' Handbook and $T$ is again the gas temperature (K).

Several methods are available to calculate the viscosity of a gas mixture. Equation \ref{eq:graham} calculates the mixture viscosity from the sum of the mole fraction and viscosity product of each gas component in the mixture \cite{Graham-1846} while Equation \ref{eq:herning} accounts for the molecular weight of each gas component \cite{Herning-1936}.

\begin{equation}\label{eq:graham}
    \mu_{mix} = \sum(x_i \cdot \mu_i)
\end{equation}

\begin{equation}\label{eq:herning}
    \mu_{mix} = \frac{\sum(\mu_i \cdot x_i \cdot \sqrt{MW_i})}{\sum(x_i \cdot \sqrt{MW_i})}
\end{equation}

\subsection{Fluidization correlations}

For a bed of particles, the minimum fluidization velocity $U_{mf}$ is the gas velocity at which the drag force of the upward moving gas equals the weight of the particles. Kunii and Levenspiel \cite{Levenspiel-1991} provide the following equation for calculating minimum fluidization velocity

\begin{equation}
    U_{mf} = \frac{Re_{p,mf} \mu}{d_p \rho_g}
\end{equation}

\noindent where $\mu$ is gas viscosity (kg/m\,s), $d_p$ is particle diameter (m), $\rho_g$ is gas density (kg/m$^3$), and $Re_{p,mf}$ is the particle Reynolds number (-) at minimum fluidization conditions.

The Reynolds number is calculated from the Archimedes number ($Ar$) and two dimensionless constants ($a$, $b$) which represent experimental coefficients.

\begin{equation}
    Re_{p,mf} = \left( a^2 + b Ar \right)^{1/2} - a
\end{equation}

\begin{equation}
    Ar = \frac{d_p^3 \rho_g (\rho_s - \rho_g) g}{\mu^2} \qquad
    a = \frac{K_2}{2 K_1} \qquad
    b = \frac{1}{K_1}
\end{equation}

\noindent The constants $K_1$ and $K_2$ are determined from the following equations which are based on the Ergun pressure drop equation for a bed of particles

\begin{equation}
    K_1 = \frac{1.75}{\epsilon_{mf}^3 \phi} \qquad
    K_2 = \frac{150(1-\epsilon_{mf})}{\epsilon_{mf}^3 \phi^2}
\end{equation}

\noindent where $\epsilon_{mf}$ is the bed void fraction (-) at minimum fluidization and $\phi$ is sphericity (-) of the bed particles. For this paper, three correlations based on the work of Ergun, Wen and Yu, and Grace were used to calculate $U_{mf}$. The Ergun approach applied the void fraction $\epsilon_{mf}$ and particle sphericity $\phi$. The Wen and Yu approach utilized the experimental coefficients of $a = 33.7, b = 0.0408$ while the Grace approach used $a = 27.2, b = 0.0408$.

\subsection{Pyrolysis kinetics}

A pyrolysis kinetics scheme based on the work of Di Blasi was implemented to predict the conversion of biomass into gas, tar, and char products \cite{Blasi-1993,Blasi-2001}. Figure \ref{fig:blasi} gives an overview of the scheme and its reaction mechanisms. Reactions 1--3 represent the primary conversion of biomass while reactions 4--5 are secondary reactions that reduce tar yield at long residence times.

\begin{figure}[H]
    \centering
    \includegraphics[width=0.4\textwidth]{blasi.pdf}
    \caption{Diagram of the Di Blasi pyrolysis kinetics scheme for conversion of biomass to gas, tar, and char products.}
    \label{fig:blasi}
\end{figure}

The pyrolysis reactions were modeled as first-order Arrhenius type equations where the rate constant $k$ for each reaction is given as

\begin{equation}
    k = A\,e^{E / RT}
\end{equation}

\noindent such that $A$ is the pre-factor, $E$ is the activation energy, $R$ is the gas constant, and $T$ is the reaction temperature. Kinetic parameters for each reaction are listed in Table \ref{tab:kinetic-params}.

\begin{table}[H]
    \centering
    \caption{Kinetic parameters for the Di Blasi biomass pyrolysis scheme.}
    \begin{tabular}{cllc}
        \hline
        Reaction    & A (1/s)               & E (kJ/mol)    & Reference     \\
        \hline
        1           & $4.38 \times 10^9$    & 152.7         & \cite{Blasi-2001} \\
        2           & $3.27 \times 10^6$    & 111.7         & \cite{Blasi-2001} \\
        3           & $1.08 \times 10^{10}$ & 148.0         & \cite{Blasi-2001} \\
        4           & $4.28 \times 10^6$    & 108.0         & \cite{Blasi-1993} \\
        5           & $1.00 \times 10^6$    & 108.0         & \cite{Blasi-1993} \\
        \hline
    \end{tabular}
    \label{tab:kinetic-params}
\end{table}

\subsection{Parameters}

Parameters for the reduced-order model and CFD simulations are provided in Table \ref{tab:params}. Biomass particle characteristics and properties are representative of loblolly pine. Bed particle characteristics are for typical sand material. Operating conditions and reactor dimensions are based on the previously discussed NREL 2FBR fluidized bed pyrolysis unit.

\begin{table}[H]
    \centering
    \caption{Biomass, bed, and reactor modeling parameters. Particle diameters represent the Sauter-mean diameter.}
    \begin{tabular}{lrll}
        \hline
        Parameter & Value & Units & Description \\
        \hline
        d$_\textrm{p,\,bed}$            & 235   & \textmugreek m & diameter of bed Particle      \\
        \straightphi$_\textrm{bed}$     & 0.0   &                & sphericity of bed particle    \\
        d$_\textrm{p,\,bio}$            & 135   & \textmugreek m & diameter of biomass particle  \\
        \straightphi$_\textrm{bio}$     & 0.0   &                & sphericity of biomass particle\\
        \textrho$_\textrm{bio}$         & 540   & kg/m$^3$       & density of biomass particle   \\
        h$_\textrm{reactor}$            & 43.18 & cm             & reactor height                \\
        h$_\textrm{static}$             & 10.16 & cm             & static bed height             \\
        T                               & 773   & K              & reactor temperature           \\
        \hline
    \end{tabular}
    \label{tab:params}
\end{table}

\subsection{Simulation cases}

Table \ref{tab:cases} represents the CFD simulations conducted for this paper. Each row is for a different simulation case which is performed for a particular gas composition.

\begin{table}[H]
    \centering
    \caption{Simulation cases for different gas mixtures where columns denote gas percentage.}
    \begin{tabular}{crrrrrr}
        \hline
        Case    & N$_2$ & H$_2$     & H$_2$O    & CO    & CO$_2$    & CH$_4$    \\
        \hline
        1       & 100   & 0         & 0         & 0     & 0         & 0         \\
        2       & 0     & 100       & 0         & 0     & 0         & 0         \\
        3       & 0     & 0         & 100       & 0     & 0         & 0         \\
        4       & 0     & 0         & 0         & 100   & 0         & 0         \\
        5       & 0     & 0         & 0         & 0     & 100       & 0         \\
        6       & 0     & 0         & 0         & 0     & 0         & 100       \\
        7       & 20    & 20        & 0         & 20    & 20        & 20        \\
        8       & 50    & 0         & 0         & 0     & 50        & 0         \\
        9       & 50    & 0         & 0         & 50    & 0         & 0         \\
        10      & 0     & 0         & 50        & 50    & 0         & 0         \\
        11      & 100   & 0         & 0         & 0     & 0         & 0         \\
        12      & 80    & 20        & 0         & 0     & 0         & 0         \\
        13      & 60    & 40        & 0         & 0     & 0         & 0         \\
        14      & 50    & 50        & 0         & 0     & 0         & 0         \\
        15      & 40    & 60        & 0         & 0     & 0         & 0         \\
        16      & 30    & 70        & 0         & 0     & 0         & 0         \\
        17      & 20    & 80        & 0         & 0     & 0         & 0         \\
        18      & 15    & 85        & 0         & 0     & 0         & 0         \\
        19      & 10    & 90        & 0         & 0     & 0         & 0         \\
        20      & 5     & 95        & 0         & 0     & 0         & 0         \\
        21      & 0     & 100       & 0         & 0     & 0         & 0         \\
        \hline
    \end{tabular}
    \label{tab:cases}
\end{table}
