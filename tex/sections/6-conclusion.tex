%!TEX root = ../main.tex

\section{Conclusion}

Gas effects in a fluidized bed biomass pyrolysis reactor using engineering correlations, low-order models, and CFD simulations were investigated for N2, H2, H2O, CO, CO2, and CH4 carrier gas mixtures. Our findings reveal viscosity of a gas mixture can be significantly underestimated depending on the model. Furthermore, fluidization characteristics such as U$_\text{mf}$ are greatly affected by gas properties but the effect on biomass pyrolysis yields is negligible.

From our simulations, we found that the yield of bio-oil is independent of the carrier gas mixture when the flowrate is constant, with the average the tar yields varying less than 2\% across each of the mixtures investigated. Furthermore, by utilizing a low molecular weight gas such as H$_2$ while maintaining a constant U$_\text{s}$/U$_\text{mf}$, the bio-oil yields can be increased 7.7\%. This is due to the lower density H$_2$ producing similar hydrodynamics as N$_2$ at higher gas flowrates. These higher flowrates result in shorter gas residence times, and as a result, less secondary reactions converting bio-oil to light gasses and char.   
