%!TEX root = ../main.tex

\section*{Abstract}

Fast pyrolysis of biomass in a fluidized bed reactor is typically conducted in a nitrogen gas environment. Recycling product gas can improve the economics of operating such a system by reducing reliance on pure process streams, but much less is known about how recycling pyrolysis product gas may affect fluidization behavior and pyrolysis kinetics. Therefore, gas effects in a fluidized bed biomass pyrolysis reactor were investigated using engineering correlations, low-order models, and CFD simulations for N$_2$, H$_2$, CO, CO$_2$, and CH$_4$ carrier gas mixtures. Our findings reveal viscosity of a gas mixture can be significantly underestimated depending on the model and correlation. Furthermore, fluidization characteristics such as U$_\textrm{mf}$ and gas-solid convective heat transfer can be greatly affected by the gas properties. By utilizing a low molecular weight gas such as H$_2$ while maintaining a constant fluidization number (U$_\text{s}$/U$_\text{mf}$), the bio-oil yields can be increased $\sim$5\%. This is due to the lower density H$_2$ producing similar hydrodynamics as N$_2$ at higher gas flow rates. These higher flow rates result in shorter gas residence times, and as a result, less secondary reactions that convert bio-oil to light gases and char.
