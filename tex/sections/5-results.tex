%!TEX root = ../main.tex

\section{Results and discussion}

This section provides results and related discussions for the effects of different fluidization gases on the operation and conversion of a bubbling fluidized bed reactor.

\subsection{Comparison of gas properties}

Molecular weight, viscosity, density, thermal conductivity, heat capacity, and Prandtl number of the individual gases investigated in this paper are shown in Figure \ref{fig:gas-properties}. The gas properties were calculated at a pressure of 101,325 Pa and a temperature of 773.15 K (500$^\circ$C). The lightest gas in terms of molecular weight and density is hydrogen while the heaviest gas is carbon dioxide. The highest viscosity is noted for the nitrogen gas while hydrogen has the lowest viscosity. The largest thermal conductivity is for hydrogen at approximately 0.36 W/(m\,K) while the other gases remain below 0.12 W/(m\,K). The highest heat capacity is obtained for methane at 62 J/(mol\,K) while the lowest is for hydrogen at 29 J/(mol\,K). The Prandtl number is similar for all the gases except for water vapor.

\begin{figure}[H]
    \centering
    \includegraphics[width=\textwidth]{gas-properties.pdf}
    \caption{Comparison of molecular weight (MW), viscosity ($\mu$), density ($\rho$), thermal conductivity (k), heat capacity (Cp), and Prandtl number (Pr) for each gas at 101,325 Pa and 773.15 K (500$^\circ$C).}
    \label{fig:gas-properties}
\end{figure}

Properties for molecular weight, viscosity, and density for the gas mixtures investigated in this paper are shown in Figure \ref{fig:mix-properties}. Similar to the individual gas properties, the mixture properties were calculated at 101,325 Pa and 773.15 K (500$^\circ$C). The fraction of each gas in the mixture is given by the values shown at the top of each column in the figure. For example, the hydrogen and nitrogen mixture is comprised of 80\% hydrogen and 20\% nitrogen which is labeled as $0.8 + 0.2$. As expected, the carbon dioxide mixture is the heaviest in terms of molecular weight and density.

\begin{figure}[H]
    \centering
    \includegraphics[width=\textwidth]{mix-properties.pdf}
    \caption{Comparison of gas mixture properties for molecular weight, viscosity, and density at 101,325 Pa and 773.15 K. Fraction of each gas component is shown at the top of each column.}
    \label{fig:mix-properties}
\end{figure}

\subsection{Fluidization effects}

Minimum fluidization velocity (Umf) of the bed material for the different fluidization gases is presented in Figure \ref{fig:gas-umf}. The hydrogen gas requires about twice the gas velocity to fluidize the sand bed compared to the nitrogen gas.

\begin{figure}[H]
    \centering
    \includegraphics[width=0.8\textwidth]{gas-umf.pdf}
    \caption{Comparison of minimum fluidization velocity (Umf) for different fluidization gases. Values calculated with the Ergun, Grace, Richardson, and Wen and Yu correlations.}
    \label{fig:gas-umf}
\end{figure}

\subsection{Evaluation of the kinetic scheme}

The Di Blasi kinetics were put to use in a batch reactor model to investigate the time scales associated with the reaction mechanisms. Figure \ref{fig:batch-blasi} is an overview of the biomass conversion and product yields using the Di Blasi kinetics in a batch reactor at 773.15 K (500$^\circ$C). At this temperature, without the effects of secondary reactions, the kinetics offer a maximum achievable tar yield of 78\% within 5 seconds. However, if secondary reactions occur during the entire pyrolysis process then a maximum tar yield of only 53\% is possible. The Di Blasi kinetics suggest that minimizing the extent of secondary reactions is critical to producing the maximum possible tar yield.

A range of reaction temperatures were applied to the Di Blasi kinetics in the batch reactor model as shown in Figure \ref{fig:batch-blasi-temps}. The kinetics suggest that temperature has a neglible effect on primary tar yield but effects of secondary reactions are more pronounced. When secondary reactions occur during the entire pyrolysis process, maximum tar yields are realized at higher temperatures but with shorter residence times. These results suggest that if secondary reactions are minimized then temperature should not have a drastic effect on tar yield.

\begin{figure}[H]
    \centering
    \includegraphics[width=\textwidth]{batch-blasi.pdf}
    \caption{Biomass conversion and product yields in a batch reactor model at 773.15 K (500$^\circ$C) according to the Di Blasi kinetic reactions. Results shown for primary reactions only (left) along with primary and secondary reactions (right).}
    \label{fig:batch-blasi}
\end{figure}

\begin{figure}[H]
    \centering
    \includegraphics[width=\textwidth]{batch-blasi-temps.pdf}
    \caption{Tar yields for reaction temperatures of 753.15--853.15 K (480--580$^\circ$C) using the Di Blasi kinetics in a batch reactor model. Results shown for primary tar (left) along with primary and secondary tar (right).}
    \label{fig:batch-blasi-temps}
\end{figure}

\subsection{CFD-DEM validation}

The predicted yield of pyrolysis products (bio-oil, light gas, and biochar) was validated against experimental data reported by [XXX]. In their experimental work, [XXX] carried out biomass pyrolysis in the same NREL 2FBR fast pyrolysis system that is modeled and simulated in this research. Additionally, the process variables used in the experimental work are consistent with those implemented for the N2 and H2 cases in this research. Figure XXX shows that the predicted yields of pyrolysis products closely follow the experimental data with absolute deviation ranging between 1\% and 6\%. The largest observed deviations occur in the prediction of bio-oil and are attributed to the non-closure of mass balance for the experimental data. The reported mass closure for the experimental data was about 94\%. A mass-proportional adjustment of the experimental data to enforce 100\% mass closure decreases the absolute deviation of bio-oil prediction to about 2\% or less.

From a qualitative point of view, the implemented CFD-DEM simulation in this research was able to acceptably predict the increase in light gas yield and decrease in biochar yield when fluidizing gas was changed from N2 to H2, as seen in the experimental data. Predicted bio-oil yield slightly increased when fluidizing gas was changed from N2 to H2, contrary to experimental data showing a slight decrease. The relative change in bio-oil yield between N2 to H2 was however quite small for both experimental data (2\%) and CFD-DEM prediction (-1\%).

These results demonstrate that the CFD-DEM model implemented in this research is capable of realistically simulating the characteristic effects of fluidizing gas on the performance of lignocellulosic biomass pyrolysis.

% FIGURE HERE

\subsection{Fluidizing gas effect on pyrolysis performance}

Figure XXX presents the volume-time averaged pressure drop and temperature along the height of the fluidized bed reactor. The different fluidizing gases considered in this research demonstrated similar effects on the pressure drop profile along the reactor height. Overall, the averaged bed height – as evidenced by the inflection point on the pressure drop curve – was about 0.14 m, regardless of fluidized gas. Similarly, the total pressure drop across the reactor was consistently about 1440 Pa for all fluidizing gases and mixtures. The volume-time averaged gas temperature ranged between 495$^\circ$C and 500$^\circ$C, depending on position along the reactor height and fluidizing gas. Gas temperature generally dipped around the biomass inlet and at the dense-bed/dilute- phase interface. The most noticeable trend in gas temperature occurs in the dilute-phase, with increasing gas temperature along the height of the reactor. Also noteworthy is the fact that gas temperature in the dilute-phase was highest when H2 was used as fluidizing gas. This observation is attributable to the large difference in the thermal conductivity of H2 and the other fluidizing gases (Figure XXX). The impact of the difference in the thermal conductivity of fluidizing gases is also evident in the average particle temperature and mass loss profile (Figure XXX). When H2 was used as fluidizing gas, biomass particles experienced significantly higher heating rate, and consequently higher mass loss rate, compared to when other fluidizing gases were used. Biomass heatingandmasslossratefollowtheorder:H2 >CH4 >H2O>CO2 >N2 +CO2 >N2 >N2 +CO> CO, irrespective of the initial size of the biomass particle.

Furthermore, it was observed that tar conversion reactions (Reactions 4 and 5) slightly changed among fluidizing gas used, with the lowest being N2 and the highest being H2 (Figure 5). This observation explains the reason why despite H2 yielded the highest particle heating and mass loss rate (Figure XXX), and one of the longest residence times (Figure XXX), its bio-oil yield relative to biomass flow rate is negligibly different from the bio-oil yield with other fluidizing gases, especially N2. Nevertheless, the fact that we found that fluidizing gas can notably increase biomass heating and mass loss rate (pyrolysis conversion rate) suggest potential process intensification implication because increased heating and pyrolysis rate represents a system where pyrolysis can be completed at an increased rate and consequently offering increased system throughput. Our finding suggests that, at the least, fluidizing gas with produced light gases can be recirculated as fluidizing gas without detrimental consequences on pyrolysis performance.

% FIGURES HERE
