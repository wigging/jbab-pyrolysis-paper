%!TEX root = ../main.tex

\section{Introduction}

Fast pyrolysis is a versatile method for thermochemical conversion of solid biomass into liquid bio-oil which can be used for bio-fuel and high-value chemical production. Bio-oil is commonly generated in bubbling fluidized bed and circulating fluidized bed reactor systems in which biomass particles rapidly devolatilize in the absence of oxygen into mixtures of light gases, condensable bio-oil vapors, and solid char \cite{Bridgwater-1999, Bridgwater-2018a, Mohan-2006}. Since biomass pyrolysis normally occurs in a non-oxidizing environment, the fluidization gas (carrier gas) is often pure nitrogen \cite{Mohan-2006}. To maximize bio-oil yields, the reactor typically operates at temperatures near 500$^\circ$C and must maintain particle residence times up to 10 seconds and gas residence times less than 2 seconds \cite{Bridgwater-2018a}. Deviations from these conditions can result in significant production and quality penalties; therefore, optimal reactor design and control become crucial to achieving commercially viable bio-oil production.

To improve the economic feasibility of biomass fast pyrolysis systems, char can be burned for process heat while recycled pyrolysis gas can assist with fluidization \cite{Bridgwater-1999, Mante-2012, Elkasabi-2015}. The major generated components of pyrolysis gas are CO, CO$_2$, CH$_4$, H$_2$, along with other light hydrocarbons \cite{Asadullah-2008, Zhang-2011}. Several experiments investigated the effects of these gases on reactor conditions and pyrolysis yields \cite{Mante-2012, Mullen-2013, Zhang-2011, Elkasabi-2015} but modeling the effects of the different gases was not discussed.

Autothermal pyrolysis experiments in a fluidized bed reactor has shown that the presence of oxygen in the carrier gas can prevent reactor clogging by reducing char formation \cite{Kim-2014}. The addition of oxygen can also improve heat transfer within the reactor via partial oxidiation of the pyrolysis products without significant decreases in bio-oil yield \cite{Polin-2019a}. Substituting air for nitrogen gas allowed for higher superficial velocities which promoted elutriation of char from reactor experiments while having neglible effect on bio-oil yield \cite{Polin-2019b}. Modeling the fluidization of the autothermal experiments was not discussed in the available literature.

There are several fluidized bed reactor models that investigate the hydrodynamics and conversion of biomass at fast pyrolysis conditions \cite{Papadikis-2009, Papadikis-2010, Mellin-2014, Xiong-2016, Xue-2011}. These models assume the carrier gas is pure nitrogen which is a typical scenario for biomass fast pyrolysis. We are not aware of any models in the biomass pyrolysis literature that investigate the effects of a carrier gas other than pure nitrogen. Consequently, our objective in this paper is to evaluate different fluidization gases and their effects on the hydrodynamics and biomass conversion in a bubbling fluidized bed reactor operating at fast pyrolysis conditions. Our methodology uses engineering correlations, reducing-order modeling techniques, and CFD simulations to model these effects and compare the model results (where applicable) to experimental data.
