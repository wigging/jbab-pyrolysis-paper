%!TEX root = ../main.tex

\section{Introduction}

Fast pyrolysis is a versatile method for thermochemical conversion of solid biomass into liquid bio-oil which can be used for bio-fuel and high-value chemical production. Bio-oil is commonly generated in bubbling fluidized bed and circulating fluidized bed reactor systems in which biomass particles rapidly devolatilize in the absence of oxygen into mixtures of light gases, condensable bio-oil vapors, and solid char \cite{Bridgwater-1999, Bridgwater-2018a, Mohan-2006}. Since biomass pyrolysis normally occurs in a non-oxidizing environment, the fluidization gas (carrier gas) is often pure nitrogen \cite{Mohan-2006}. To maximize bio-oil yields, the reactor typically operates at temperatures near 500$^\circ$C and must maintain particle residence times up to 10 seconds and gas residence times less than 2 seconds \cite{Bridgwater-2018a}. Deviations from these conditions can result in significant production and quality penalties, therefore optimal reactor design and control become crucial to achieving commercially viable bio-oil production.

To improve the economic possibilities of biomass fast pyrolysis systems, char can be burned for process heat while recycled pyrolysis gas can assist with fluidization \cite{Bridgwater-1999, Mante-2012}. The major generated components of pyrolysis gas are CO, CO$_2$, CH$_4$, H$_2$, and other light hydrocarbons \cite{Asadullah-2008, Zhang-2011}. Several experiments investigated the effects of these gases on reactor conditions and pyrolysis yields \cite{Mante-2012, Mullen-2013, Zhang-2011} but modeling the effects of the different gases was not discussed.

There are several models available that investigate the hydrodynamics and conversion of biomass at fast pyrolysis conditions in fluidized bed reactors \cite{Papadikis-2010, Mellin-2014}. As is typical for biomass pyrolysis, these models assume the fluidization gas is pure nitrogen. The authors are not aware of any published models in the biomass pyrolysis literature that account for the effects of fluidization or carrier gas other than nitrogen.

This paper uses engineering correlations, reduced-order modeling techniques, and CFD simulations to investigate the effects of gas mixtures in a fluidized bed biomass pyrolysis reactor. The scope of this study is to evaluate different gas mixtures and there effects on the hydrodynamics and biomass conversion in fluidized bed reactors operating at fast pyrolysis conditions.
