%!TEX root = ../main.tex

\section{Conclusion}

Gas effects in a fluidized bed biomass pyrolysis reactor using engineering correlations, low-order models, and CFD simulations were investigated for N$_2$, H$_2$, H$_2$O, CO, CO$_2$, and CH$_4$ carrier gas mixtures. Our findings reveal the importance of evaluating models and correlations for determining properties of a gas mixture. Two of the gas mixture models available in the literature (Brokaw as well as Herning \& Zipperer) compared well to experimental viscosity data. This contradicts results reported by Davidson who does not recommend the Herning and Zipperer method for hydrogen gas mixtures. The models from Davidson, Graham, and Wilke significantly underestimate the viscosity of the hydrogen mixtures investigated in this paper.

Fluidization characteristics and gas-solid heat transfer in a bubbling fluidized bed reactor can be greatly affected by the carrier gas properties. Based on our results, hydrogen gas requires approximately twice the flow rate of nitrogen gas to reach similar fluidization conditions in a BFB reactor. We also found that the thermal properties of the hydgrogen gas improves its convective heat transfer capability thus increasing its potential to pyrolyze biomass compared to a nitrogen gas environment. From our simulations, we found that the yield of bio-oil is independent of the carrier gas mixture when the flow rate is constant, with the average tar yield varying less than 2\% across each of the mixtures investigated. Furthermore, by utilizing a low molecular weight gas such as H$_2$ while maintaining a constant U$_\text{s}$/U$_\text{mf}$, the bio-oil yields can be increased $\sim$5\%. This is due to the lower density H$_2$ producing similar hydrodynamics as N$_2$ at higher gas flowrates. These higher flowrates result in shorter gas residence times, and as a result, less secondary reactions converting bio-oil to light gases and char.
