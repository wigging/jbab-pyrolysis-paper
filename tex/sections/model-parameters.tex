%!TEX root = ../main.tex

\section{Model parameters}

Particle size distribution of the biomass feedstock along with the mass flow rate associated with each size bin is given in Table \ref{tab:params-part-size}. The initial biomass chemical composition used for the Di Blasi kinetics model is shown in Table \ref{tab:params-chem-biomass}. Other parameters related to the biomass and sand particles along with reactor operation and simulation settings are provided in Table \ref{tab:params}. Biomass particle characteristics and properties are representative of loblolly pine while the bed particle characteristics are for a typical sand material. Operating conditions and reactor dimensions are based on the previously discussed NREL 2FBR fluidized bed pyrolysis unit.

\begin{table}[H]
    \centering
    \caption{Particle size distribution for the biomass feedstock.}
    \label{tab:params-part-size}
    \begin{tabular}{>{\centering}p{2.5cm} >{\raggedleft}p{2.2cm} >{\raggedleft\arraybackslash}p{2.5cm}}
        \toprule
        Sauter mean diameter (\textmugreek m) & Mass fraction (\%) & Mass flow rate (kg/hr) \\
        \midrule
        278 & 12.1 & 0.018 \\
        344 & 51.0 & 0.076 \\
        426 & 34.2 & 0.051 \\
        543 & 2.7  & 0.004 \\
        \bottomrule
    \end{tabular}
\end{table}

\begin{table}[H]
    \centering
    \caption{Initial chemical composition of the biomass feedstock for the Di Blasi kinetics model.}
    \label{tab:params-chem-biomass}
    \begin{tabular}{lrr}
        \toprule
        Species & Mass fraction (\%) & Density (kg/m$^3$) \\
        \midrule
        moisture & 4.0  & 1,000 \\
        wood     & 95.9 & 500 \\
        ash      & 0.1  & 2,000 \\
        char     & 0.0  & 300 \\
        \bottomrule
    \end{tabular}
\end{table}

\begin{table}[H]
    \centering
    \caption{Parameters for the biomass, sand (bed material), and reactor operation. Biomass C$_\text{p}$ calculated from particle composition. Parameters for simulation settings also given.}
    \label{tab:params}
    \begin{tabular}{lll}
        \toprule
        Parameter & Value & Description \\
        \midrule
        biomass particle \\
        e$_\text{p}$    & 0.2        & particle-particle coefficient of restitution \\
        e$_\text{w}$    & 0.2        & particle-wall coefficient of restitution \\
        e$_\text{s}$    & 0.2        & particle-sand coefficient of restitution \\
        $\mu_\text{p}$  & 0.1        & particle-particle coefficient of friction \\
        $\mu_\text{w}$  & 0.2        & particle-wall coefficient of friction \\
        $\mu_\text{s}$  & 0.1        & particle-sand coefficient of friction \\
        k$_\text{n}$    & 100 N/m    & particle spring constant \\
        \\
        sand particle \\
        d$_\text{p}$       & 453 $\mu$m          & particle diameter \\
        $\rho_\text{p}$    & 2500 kg/m$^3$       & particle density \\
        C$_\text{p}$       & 830 J/(kg\,K)      & particle heat capacity \\
        $\phi$             & 0.94                & particle sphericity \\
        e$_\text{p}$       & 0.61                & particle-particle coefficient of restitution \\
        e$_\text{w}$       & 0.61                & particle-wall coefficient of restitution \\
        $\mu_\text{p}$     & 0.1                 & particle-particle coefficient of friction \\
        $\mu_\text{w}$     & 0.2                 & particle-wall coefficient of friction \\
        k$_\text{n}$       & 100 N/m             & particle spring constant \\
        \\
        reactor operation \\
        d$_\text{inner}$      & 5.25 cm       & inner reactor diameter \\
        H$_\text{reactor}$    & 43.18 cm      & reactor height \\
        H$_\text{static}$     & 10.16 cm      & static bed height \\
        p$_\text{gas}$        & 101.325 kPa   & gas pressure \\
        T$_\text{gas}$        & 773.15 K      & gas temperature \\
        Q$_\text{gas}$        & 14 SLM        & inlet gas flowrate \\
        \\
        simulation settings \\
        $\Delta_\text{x} \times \Delta_\text{y} \times \Delta_\text{z}$ & $4.3 \times 4.4 \times 4.3$ mm & CFD cell size \\
        $\Delta_\text{x}$   & varies & time step \\
        wt$_\text{bio}$      & 10     & biomass parcel statistical weight \\
        wt$_\text{sand}$     & 20     & sand parcel statistical weight \\
        s$_\text{gas}$               & ideal  & gas phase equation of state \\
        \bottomrule
    \end{tabular}
\end{table}

Table \ref{tab:flowrates} summarizes the CFD simulations conducted for this study. Each row represents a different simulation case that was performed for a particular gas composition. An additional $2.83\times10^{-5}$ m$^3$/s of N$_2$ at 500$^\circ$C was supplied at the fluidizing gas inlet and $2.55\times10^{-5}$ m$^3$/s of N$_2$ at 25$^\circ$C was supplied at the biomass feed inlet for all cases. For cases 1--8, the total flow rate was kept constant, resulting in a constant superficial gas velocity inside the reactor. For cases 9--13, the gas flow rates were modified to maintain a constant U/U$_\text{mf}$ = 3 in each simulation. Also, cases 9--13 represent H$_2$ mass fractions of 0.2, 0.4, 0.6, 0.8 and 1. Pure hydrogen is modeled because of its drastically different properties compared to the other gases (see following section); however, its applicability in experiments is limited due to safety concerns.

% I updated the flowrate used for N2 in Case 1 and Case 8 to exclude the constant 2.83e-5 m3/s flowrate of N2 since the remaining entries did not include this portion of the flow and it was confusing

\begin{table}[H]
    \centering
    \caption{CFD-DEM simulation cases for different gas mixtures. Columns denote gas flow rate in m$^3$/s at 500$^\circ$C.}
    \begin{tabular}{llrrrrrr}
        \toprule
        ID & Case                    & N$_2$     & H$_2$    & H$_2$O   & CO       & CO$_2$    & CH$_4$   \\
        \midrule
        1  & N$_2$                   & 6.37e-04  & 0        & 0        & 0        & 0         & 0        \\
        2  & H$_2$                   & 0         & 6.37e-04 & 0        & 0        & 0         & 0        \\
        3  & CO                      & 0         & 0        & 0        & 6.37e-04 & 0         & 0        \\
        4  & CO$_2$                  & 0         & 0        & 0        & 0        & 6.37e-04  & 0        \\
        5  & CH$_4$                  & 0         & 0        & 0        & 0        & 0         & 6.37e-04 \\
        6  & 0.5 N$_2$ + 0.5 CO      & 3.18e-04  & 0        & 0        & 3.18e-04 & 0         & 0        \\
        7  & 0.5 N$_2$ + 0.5 CO$_2$  & 3.18e-04  & 0        & 0        & 0        & 3.18e-04  & 0        \\
        8  & 0.22 N$_2$ + 0.78 H$_2$ & 1.42e-04  & 4.95e-04 & 0        & 0        & 0         & 0        \\
        9  & 0.10 N$_2$ + 0.90 H$_2$ & 6.17e-05  & 5.75e-04 & 0        & 0        & 0         & 0        \\
        10 & 0.05 N$_2$ + 0.95 H$_2$ & 2.89e-05  & 6.08e-04 & 0        & 0        & 0         & 0        \\
        11 & 0.02 N$_2$ + 0.98 H$_2$ & 1.12e-05  & 6.26e-04 & 0        & 0        & 0         & 0        \\
        12 & H$_2$                   & 0         & 6.37e-04 & 0        & 0        & 0         & 0        \\
        \bottomrule
    \end{tabular}
    \label{tab:flowrates}
\end{table}
